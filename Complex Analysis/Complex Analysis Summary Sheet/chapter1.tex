\setcounter{chapter}{-1}
\chapter{Introduction}
\setcounter{section}{2}
\section{Basic notations}
\begin{definition}
    We say that $U \subset C$ is \textit{open} if for every $z \in U$ there is some $\varepsilon  > 0 $ such that an open ball $B(z, \varepsilon)  \defined \{w \in \C : |z - w| < \varepsilon\} \subset U$. Any open set $U$ containing $z$ is called a \textit{neghborhood} of $z$. 
\end{definition}
\begin{definition}
    A connected open subset $D \subseteq \C$ of the complex plane will be called a \textit{domain}.
\end{definition}
\chapter{Complex Differentiability}
\section{Complex Differentiability}
\begin{definition}(Complex Differentiability)
    
\end{definition}

\begin{eg}
    Suppose we have a functor.
    If $G_X \not\cong G_Y$, then  $X$ and  $Y$ are not homeomorphic.
    If `shadows' are different, then objects themselves are different too.
\end{eg}
\begin{explanation}
    Suppose $X$ and $Y$ are homeomorphic.
    Then $\exists f: X \to  Y$ and $g: Y \to  X$, maps (maps are always continuous in this course), such that $g  \circ f = 1_X$ and $f \circ g = 1_Y$.
    Then $f_*: G_X \to G_Y$ and $g_*: G_Y \to  G_X$ such that $(g \circ f)_* = (1_X)_*$ and  $(f \circ g)_* = (1_Y)_*$. Using the rules discussed previously, we get
    \[
    g_*  \circ f_* = 1_{G_X} \quad f_*  \circ  g_* = 1_{G_Y}
    ,\] 
    which means that $f_* : G_X \to  G_Y$ is an isomorphism.
\end{explanation}
