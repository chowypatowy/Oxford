\setcounter{chapter}{2}
\chapter{Cauchy's Formula and its Applications}
\section{Cauchy's Integral Formula}
\begin{theorem}[Cauchy's Integral Formula]
    Suppose that $f : U \to \C$ is a holomorphic function on an open set $U$, $w \in  U$ and $\gamma $ is a simple positively oriented closed curve such that $\gamma ^*$ and the interior of $\gamma $ are inside of $U$. Then for all $w$ that are inside of $\gamma $ w have
    \[
    f(w) = \frac{1}{2\pi i} \int_{\gamma }^{} \frac{f(z)}{z-w}dz. 
    \] 
\end{theorem}
\begin{remark}
    Same result holds for any oriented curve $\gamma $ once we weigh the LHS by the winding number of a path around the point $w \not\in \gamma^*$, provided that $f$ is holomorphic on the inside of $\gamma $.
\end{remark}
\begin{corollary}[Cauchy Formula for multiple curves]
    Let $U$ be a bounded domain with piecewise $C^{1}$ boundary which has finitely many components and $f$ be a function holomorphic in the closure of $U$ (this means that it is holomorphic in some open domain that contains the closure of $U$). We parametrize each boundary component of $U$ by a contour $\gamma _i$ in such way that $i \gamma _i'(t)$ is an inward normal. This means that the 'outer' boundary is positively oriented (i.e. counter-clockwise) and all 'inner' components are negatively oriented (i.e. clockwise). Denoting $\int_{\delta U}^{}  = \sum \int_{\gamma _i}^{}  $ we have
\[
    \int_{\delta U} f(z)dz = 0
\] and
\[
    \frac{1}{2\pi i}\int_{\delta U} \frac{f(z)}{z-w}dz = f(w), w \in  U.
\] 
    
\end{corollary}

\section{Homotopy Version of Cauchy's Theorem}
\textbf{NON-EXAMINABLE}
\section{Applications of the Integral Formula}

\begin{corollary}
    If $f : U \to \C$ is holomorphic on an open set $U$, then for any $z_0 \in  U, f(z)$ is equal to its Taylor series at $z_0$ and the Taylor series converges on any open disk centered at $z_0$ lying in $U$. Moreover the derivatives of $f$ at $z_0$ are given by
    \[
        f^{(n)}(z_0) = \frac{n!}{2\pi i}\int_{\gamma (z_0, r)} \frac{f(z)}{(z-z_0)^{n+1}} dz
    \] 
\end{corollary}
\begin{definition}    
    A function which is locally given by a power series is said to be \textit{analytic}. We have thus shown that any holomorphic function is actually analytic, and from now on we may use the terms interchangeably.
\end{definition}
\begin{definition}
    A function $f: \C \to \C$ is \textit{entire} if it is complex differentiable on the whole complex plane.  
\end{definition}
\begin{theorem}[Liouville]
    Let $f : \C \to  \C$ be an entire function. If $f$ is bounded then it is constant.
\end{theorem}
\begin{remark}
    Liouville's theorem is another manifestation of the unique properties of holomorphic functions. In real analysis, $f(x) = \frac{1}{1+x^2}$ is real-analytic in the entire $\R$, but it is bounded. \textbf{ This is also one of the first examples of a dichotomy which often appears in complex analysis. } In many cases objects are as good or as bad as they could be but nothing in between.  
\end{remark}
\begin{theorem}
    Suppose that $p(z) = \sum_{k=0}^{n} a_kz^k$ is a non-constant polynomial where $a_k \in  \C$ and $a_n \neq  0$. Then there is a $z_0 \in  \C$ for which $p(z_0) = 0$. 
\end{theorem}
\begin{remark}
    The crucial point of the above proof is that on term of the polynomial dominates the behavior for large $z$. All proofs of the fundamental theorem hinge on essentially this point.
\end{remark}
\begin{theorem}
    Suppose $f : U \to  \C$ is a continuous function on an open subset $U \subseteq \C$. If for any closed path $\gamma :[a, b] \to o U$ we have $\int_{\gamma }f(z)dz = 0$, then $f$ is holomorphic.
\end{theorem}

\section{The Identity Theorem}
\section{Isolated Singularities}
\begin{definition}
    Let $f : U \to \C$ be a function, where $U$ is open. We say that $z_0 \in \overline{U}$ is a \textit{regular} point of $f$ if $f$ is holomorphic at $z_0$. Otherwise we say that $z_0$ is \textit{singular}.\\
    We say that $z_0$ is an \textit{isolated singularity} if $f$ is holomorphic on $B(z_0, r) \setminus \{z_0\} $ for some  $r>0$.  
\end{definition}
\begin{definition}
    A function on an open set $U$ which has only isolated singularities all of which are poles is called a \textit{meromorphic} function on $U$. (Strictly, it is a function only defined on the complement of the poles in $U$.)  
\end{definition}
\begin{definition}
    Let $z_0$ be an isolated singularity of function $f$. We say that $z_0$ is
    \begin{itemize}
        \item A \textit{removable singularity} if there is a function $g$ holomorphic in $B(z_0, r)$ for some $r > 0$ such that such that $f(z) = g(z)$ in $B(z_0,r)\setminus \{z_0\} $.    
        \item A \textit{pole of order n} if there is a function $g$ holomorphic in $B(z_0, r)$ for some $r > 0$ such that $g(z_0) \neq  0$ and $f(z) = (z-z_0)^{-n}g(z)$ in $B(z_0, r)\setminus \{z_0\} $.  
    \end{itemize}
\end{definition}

\begin{eg}
    Let  $f(z) = \frac{\sin z}{z}$. $z_0=0$ is a removable singularity. 
\end{eg}

\begin{explanation}
    The function
    \[
    g(z) = 1 - \frac{z^2}{3!} + \frac{z^{4}}{5!}-\ldots
    \] is entire and coincides with $f$ outside of $z_0$. 
\end{explanation}

\begin{eg}
    Let $f(z) = \frac{\sin z}{z^{n+1}}$. This function has a pole of order $n$ at $z_0$. 
\end{eg}

\begin{explanation}
    It is easy to see for $z \neq  0$ we have $f(z) = \frac{g(z)}{z^{n}}$ where $g$ is the entire function from the previous example. Obviously $g(0) = 1 \neq  0$. 
\end{explanation}

\begin{eg}
    Let $f(z) = \sin(\frac{1}{z})$. This function is holomorphic in $C\setminus \{0\} $. It can be shown 0 is neither a removable singularity nor a pole, so it must be an essential singularity.
\end{eg}


\begin{theorem}(Laurent's Theorem)
    Suppose that $0 < r < R$ and
    \[
    A = A(z_0, r, R) = \{z : r < |z - z_0| < R\}
    \] 
    is an annulus centered at $z_0$. If $f : U \to \C$ is holomorphic on an open set $U$ which contains $\overline{A}$, then there exist $c_n \in  \C$ such that
    \[
    \sum_{n=-\infty}^{\infty} c_n (z-z_0)^n 
    \] 
    Th series converges for all $z \in A$ and it converges uniformly for all $z \in  A(z_0, r', R')$ where $r < r' < R' < R$. The series is called the \textit{Laurent series} of $f$. 
    Moreover, the $c_n$ are unique and are given by the following formulae:
    \[
        c_n = \frac{1}{2\pi i} \int_{\gamma _s} \frac{f(z)}{(z-z_0)^{n+1}}dz,
    \] 
    where $s \in [r, R]$ and for any $s > 0$ we set $\gamma _s(t) = z_0 + se^{2\pi it}$.
\end{theorem}
\begin{remark}
    Given a formula for $c_n$ in terms of  $f$, the Laurent expansion is unique. If two series converge to the same function then they coincide term-by-term.

\end{remark}
\begin{remark}
    If $f$ is holomorphic in $B(z_0, R)$, then for $n < 0$ the integrand in the formula for $c_n$ is holomorphic, hence $c_n = 0$ for all $n < 0$. For $n \ge 0$ formulas for $c_n$ are exactly the same as in Taylor's theorem so in this case the Laurent series is the same as the Taylor series.
    
\end{remark}

\begin{corollary}
    If $f : U \to \C$ is a holomorphic function and $z_0$ is an isolated singularity, then $f$ has a Laurent expansion on punctured disc $B(z_0, R)\setminus \{z_0\} $ for any $R$ such that $B(z_0, R)\setminus \{z_0\} \subset U$.  
\end{corollary}

\begin{definition}
    Let $z_0$ be an isolated singularity of $f$ and $\sum_{}^{} c_n (z-z_0)^n $ be its Laurent expansion. 
 Its \textit{principal part} of $f$ at $z_0$ is the sum of terms with negative powers and denoated $P_{z_0}f$. Namely,
\[
    P_{z_0}f = \sum_{n=-\infty}^{\infty} c_n (z-z_0)^n = \sum_{n=1}^{\infty} c_{ -n } (z-z_0)^n 
\] 
\end{definition}
\begin{proposition}
    Th principal part of $f$ at $z_0$ converges on $\C\setminus \{z_0\} $ and converges uniformly on $\C\setminus B(z_0, r)$. 
\end{proposition}
    


\begin{definition}
    Let $z_0$ be an isolated singularity of $f$. Then the \textit{residue} of $f$ at $z_0$ is defined as the coefficient $c_{-1}$ of the Laurent expansion and denoted by $Res_{z_0}f$ or $Res(f, z_0)$. 
    
\end{definition}

\begin{theorem}
    Let $z_0$ be an isolated singularity of $f$. Let $\sum _{-\infty}^{\infty} c_n(z-z_0)^n$ be its Laurent expansion. Then $z_0$ is
    \begin{itemize}
        \item A removable singularity if $c_n = 0$ for all $n < 0$. Equivalently, the principal part vanishes.
        \item A pole of order $n$ if $c_{-n} \neq  0$ and $c_k = 0$ for all $k < -n$. Equivalently, the principal part is non-trivial but contains only a finite number of non-zero terms.
        \item An essential singularity if there are arbitrarily large $n$ such that $c_{-n} \neq  0$. Equivalently, the principal part contains finitely main non-zero terms.
    \end{itemize}
\end{theorem}


\begin{theorem}(Riemann's removable singularity theorem)
    Suppose that $ U$ is an open subset of $\C$ and $z_0 \in  U$ and suppose that $f: U \setminus  \{ z_0\} \to \C$ is holomorphic. Then $z_0$ is a removable singularity if and only if $f$ is bounded near $z_0$. 
\end{theorem}
\begin{lemma}
    Let $f$ be a holomorphic function in a neighborhood of $z_0$. Then $z_0$ is a pole if and only if $|f(z)| \to \infty$ as $z \to z_0$. Moreover, in this case, the function
    \[
    h(z) = \begin{cases}
        \frac{1}{f(z)}, & z \neq z_0;\\
        0, & z=z_0;
    \end{cases}
    .\] 
    is holomorphic in a neighborhood of $z_0$ and the multiplicity of its zero at $z_0$ is equal to the order of the pole of $f$.  
\end{lemma}

\begin{theorem}(Casorati-Weierstrass)
    Let $U$ be an open subset of $\C$ and let $a \in  U$. Suppose that $f : U \setminus \{a\} \to \C$ is a holomorphic function with an isolated essential singularity at $a$. Then for all $\rho > 0$ with $B(a, \rho ) \subseteq U$, the set $f(B(a,a \rho ) \setminus \{a\}$ is dense in $\C$, that is, the closure of $f(B(a, \rho )\setminus \{a\} )$ is all of $\C$.  

\end{theorem}

\begin{theorem}(Residue Theorem)
    Suppose that $U$ is an open set in $\C$ and $\gamma $ is a closed curve that is contained in $U$ together with its inside. Suppose that $f$ is holomorphic on $U \setminus  S$ where $S$ is a finite set of isolated singularities of $f$. We also assume that $f$ has no singularities on $\gamma \star$, that is $S \cap \gamma \star = \O$. Then
    \[
        \frac{1}{2\pi i} \int_{\gamma } f(z)dz = \sum_{a \in  S} I(\gamma , a)Res_a(f)
    .\] 
    $
\end{theorem}


